% !TEX program = xelatex
%% Start of file `awesome-cv.tex'.
%% Awesome CV LaTeX Template for Samuel Calvert's Resume
%%
%% This template has been downloaded from:
%% https://github.com/posquit0/Awesome-CV
%%

\documentclass[11pt, letterpaper]{awesome-cv}

% Configure page margins with geometry package
\geometry{left=1.4cm, top=.8cm, right=1.4cm, bottom=1.8cm, footskip=.5cm}

% Specify the location of the included fonts
\fontdir[fonts/]

% Color for highlights
\colorlet{awesome}{awesome-skyblue}
% Uncomment if you would like to specify your own color
% \definecolor{awesome}{HTML}{CA63A8}

% Colors for text
\colorlet{darktext}{gray}
\colorlet{text}{darkgray}
\colorlet{graytext}{gray}
\colorlet{lighttext}{lightgray}

% Set false if you don't want to highlight section with awesome color
\setbool{acvSectionColorHighlight}{false}

% If you would like to change the social information separator from a pipe (|) to something else
\renewcommand{\acvHeaderSocialSep}{\quad\textbar\quad}

%-------------------------------------------------------------------------------
%	PERSONAL INFORMATION
%-------------------------------------------------------------------------------
% Available options: circle|rectangle,edge/noedge,left/right
% \photo{profile.png}
\name{Samuel}{Calvert}
\position{Computer Engineering Professional{\enskip\cdotp\enskip}(He/Him)}
\address{New York City Metropolitan Area}

\mobile{(302) 513-3155}
\email{hello@samuelcalvert.com}
% \homepage{www.example.com}
\github{sfcal}
\linkedin{samuel-f-calvert}
% \gitlab{gitlab-id}
% \stackoverflow{SO-id}{SO-name}
% \twitter{@twit}
% \skype{skype-id}
% \reddit{reddit-id}
% \medium{medium-id}
% \googlescholar{googlescholar-id}{name-to-display}
% \extrainfo{extra information}

\quote{Computer Engineering professional with expertise in infrastructure automation, networking, and systems troubleshooting.}

% Set the document to begin
\begin{document}

% Print the header with above personal information
\makecvheader

% Print the footer with 3 arguments(<left>, <center>, <right>)
\makecvfooter
  {\today}
  {Samuel Calvert~~~·~~~Resume}
  {\thepage}

%-------------------------------------------------------------------------------
%	CV/RESUME CONTENT
%	Each section is imported separately, open each file in turn to modify content
%-------------------------------------------------------------------------------

%-------------------------------------------------------------------------------
%	SECTION TITLE
%-------------------------------------------------------------------------------
\cvsection{Experience}

%-------------------------------------------------------------------------------
%	CONTENT
%-------------------------------------------------------------------------------
\begin{cventries}

%---------------------------------------------------------
  \cventry
    {Technical Support Engineer} % Job title
    {Haivision Network Video} % Organization
    {Remote, New York City} % Location
    {Jul. 2022 - Present} % Date(s)
    {
      \begin{cvitems} % Description(s) of tasks/responsibilities
        \item {Identify and resolve deployment problems across defense, broadcast, and enterprise clients}
        \item {Implement immutable Packer images to standarize troubleshooting procedures and cut response time by 15\%}
        \item {Troubleshoot enterprise network (500+ endpoints) and streaming infrastructure processing 50Gbps peak traffic}
        \item {Team lead on documenting critical procedures for hands-on data recovery and restore}
        \item {Optimizing streaming performance for three letter agencies' secure multicast networks}
        \item {Managed technical transition of Sony legacy RTSP streaming infrastructure to modern SRT protocol, reducing latency by 30x}
      \end{cvitems}
    }

%---------------------------------------------------------
\vspace{0.5cm}
  \cventry
    {Lab Staff}
    {University of Delaware EECIS Department}
    {Newark, DE}
    {Jun. 2019 - Aug. 2021}
    {
      \begin{cvitems}
        \item {Maintained 25 servers handling 1M+ monthly web requests and 100TB of research datasets}
        \item {Developed new ESXi based VM system for student project deployment}
        \item {Supported machine learning and AI researcher scheduling with SLURM}
        \item {Implemented automation for routine maintenance of university mail servers with Ansible}
      \end{cvitems}
    }

%---------------------------------------------------------
\vspace{0.5cm}
  \cventry
    {Research Intern}
    {University of Delaware Machine Learning}
    {Newark, DE}
    {Jun. 2020 - Aug. 2020}
    {
      \begin{cvitems}
        \item {Developed network intrusion detection systems (IDS) using GAN-based neural nets}
        \item {Simulated GNS3 virtual networks to quickly iterate over different topologies}
        \item {Containerized solution using Docker for deployment flexibility}
        \item {Created documentation and testing procedures for research outcomes}
      \end{cvitems}
    }

%---------------------------------------------------------
\end{cventries}

%-------------------------------------------------------------------------------
%	SECTION TITLE
%-------------------------------------------------------------------------------
\cvsection{Projects}

%-------------------------------------------------------------------------------
%	CONTENT
%-------------------------------------------------------------------------------
\begin{cventries}

%---------------------------------------------------------
\cventry
    {Personal Project}
    {Homelab Kubernetes Infrastructure \href{https://github.com/sfcal/homelab}{\faGithub}}
    {}
    {}
    {
      \begin{cvitems}
        \item {Designed and implemented multi-environment Kubernetes clusters running on Proxmox VMs}
        \item {Created complete infrastructure-as-code solution using Terraform, Ansible, and GitOps methodologies}
        \item {Developed automation for the entire infrastructure lifecycle from VM template creation to application deployment}
        \item {Implemented monitoring stacks with Prometheus/Grafana and self-hosted DNS solutions}
        \item {Architected environment-specific configurations while maintaining common base components}
      \end{cvitems}
    }

%---------------------------------------------------------
\end{cventries}

%-------------------------------------------------------------------------------
%	SECTION TITLE
%-------------------------------------------------------------------------------
\cvsection{Technical Skills}

%-------------------------------------------------------------------------------
%	CONTENT
%-------------------------------------------------------------------------------
\begin{cvskills}

%---------------------------------------------------------
  \cvskill
    {Infrastructure}
    {Kubernetes, Docker, Proxmox, Terraform, Ansible, GitOps}

%---------------------------------------------------------
  \cvskill
    {Programming}
    {Bash, Python, Go, HCL, Regex}

%---------------------------------------------------------
  \cvskill
    {DevOps}
    {FluxCD, Infrastructure Automation, Configuration Management}

%---------------------------------------------------------
  \cvskill
    {Networking}
    {Wireshark, Nmap, GNS3, netcat}

%---------------------------------------------------------
\end{cvskills}

%-------------------------------------------------------------------------------
%	SECTION TITLE
%-------------------------------------------------------------------------------
\cvsection{Education}

%-------------------------------------------------------------------------------
%	CONTENT
%-------------------------------------------------------------------------------
\begin{cventries}

%---------------------------------------------------------
  \cventry
    {Bachelor of Science in Computer Engineering}
    {University of Delaware}
    {Newark, DE}
    {Graduated May 2022}
    {}

\end{cventries}

%-------------------------------------------------------------------------------
%	CONTENT
%-------------------------------------------------------------------------------
\begin{cvskills}

%---------------------------------------------------------
  \cvskill
    {Languages}
    {Fluent French Speaker}

%---------------------------------------------------------
\end{cvskills}
%-------------------------------------------------------------------------------
\end{document}